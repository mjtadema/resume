%%%%%%%%%%%%%%%%%%%%%%%%%%%%%%%%%%%%%%%%%
% Twenty Seconds Resume/CV
% LaTeX Template
% Version 1.1 (8/1/17)
%
% This template has been downloaded from:
% http://www.LaTeXTemplates.com
%
% Original author:
% Carmine Spagnuolo (cspagnuolo@unisa.it) with major modifications by 
% Vel (vel@LaTeXTemplates.com)
%
% License:
% The MIT License (see included LICENSE file)
%
%%%%%%%%%%%%%%%%%%%%%%%%%%%%%%%%%%%%%%%%%

%----------------------------------------------------------------------------------------
%	PACKAGES AND OTHER DOCUMENT CONFIGURATIONS
%----------------------------------------------------------------------------------------

\documentclass[letterpaper]{twentysecondcv} % a4paper for A4

%----------------------------------------------------------------------------------------
%	 PERSONAL INFORMATION
%----------------------------------------------------------------------------------------

% If you don't need one or more of the below, just remove the content leaving the command, e.g. \cvnumberphone{}

\profilepic{me.jpg} % Profile picture

\cvname{Matthijs Tadema} % Your name
\cvjobtitle{Computational Biologist} % Job title/career

\cvdate{6 December 1993} % Date of birth
\cvaddress{Groningen, The Netherlands} % Short address/location, use \newline if more than 1 line is required
\cvnumberphone{+31 6 23 51 29 54} % Phone number
%\cvsite{http://en.wikipedia.org} % Personal website
\cvsite{}
\cvmail{m.j.tadema@pm.me} % Email address

%----------------------------------------------------------------------------------------

\begin{document}

%----------------------------------------------------------------------------------------
%	 ABOUT ME
%----------------------------------------------------------------------------------------

\aboutme{Something about me i guess} % To have no About Me section, just remove all the text and leave \aboutme{}

%----------------------------------------------------------------------------------------
%	 SKILLS
%----------------------------------------------------------------------------------------

% Skill bar section, each skill must have a value between 0 an 6 (float)
%\skills{{pursuer of rabbits/5.8},{good manners/4},{outgoing/4.3},{polite/4},{Java/0.01}}
\skills{{Python/6},{Bash/5.5},{R/2},{C/1},{GROMACS/5.5},{Linux/6},{C++/1.5},{LaTeX/5}}

%------------------------------------------------

% Skill text section, each skill must have a value between 0 an 6
%\skillstext{{lovely/4},{narcissistic/3}}

% Languages

\languages{{Dutch/native},{English/C2}}

%----------------------------------------------------------------------------------------

\makeprofile % Print the sidebar

%----------------------------------------------------------------------------------------
%	 INTERESTS
%----------------------------------------------------------------------------------------

\section{Interests}

Molecular simulation, bioinformatics, data science, open source software

%----------------------------------------------------------------------------------------
%	 EDUCATION
%----------------------------------------------------------------------------------------

\section{Education}

\begin{twenty} % Environment for a list with descriptions
	\twentyitem{since 2020}
		{Ph.D. {\normalfont in Computational Biology}}
		{University of Groningen}
		{\emph{Computationally aided protein nanopore design and engineering.}}
	\twentyitem{2017-2020}
		{M.Sc. Biomolecular sciences, magna cum laude}
		{University of Groningen}
		{Strong focus on coarse-grained molecular dynamics and computational design techniques of proteins.}
	\twentyitem{2014-2017}
		{B.Sc. Life Science and Techonlogy}
		{University of Groningen}
		{Molecular biology and enzyme engineering.}
	%\twentyitem{<dates>}{<title>}{<location>}{<description>}
\end{twenty}

%----------------------------------------------------------------------------------------
%	 PUBLICATIONS
%----------------------------------------------------------------------------------------

\section{Publications}

%\begin{twentyshort} % Environment for a short list with no descriptions
%	\twentyitemshort{JACS, 2023}{Seeing the Invisibles: Detection of Peptide Enantiomers, Diastereomers, and Isobaric Ring Formation in Lanthipeptides Using Nanopores}
%	%\twentyitemshort{<dates>}{<title/description>}
%\end{twentyshort}

\begin{twenty}
	\twentyitem{2023}
		{Translocation of linearized full-length proteins through an engineered nanopore under opposing electrophoretic force}
		{Nature Biotechnology}
		{Sauciuc, A., Morozzo della Rocca, B., Tadema, M. J., Chinappi, M., \& Maglia, G.} 
	\twentyitem{2023}
		{Seeing the Invisibles: Detection of Peptide Enantiomers, Diastereomers, and Isobaric Ring Formation in Lanthipeptides Using Nanopores}
		{JACS}
		{Abraham Versloot, R. C., Arias-Orozco, P., Tadema, M. J., Rudolfus Lucas, F. L., Zhao, X., Marrink, S. J., Kuipers, O. P., \& Maglia, G.}
	\twentyitem{2022}
		{$\beta$-Barrel Nanopores with an Acidic–Aromatic Sensing Region Identify Proteinogenic Peptides at Low pH}
		{ACS Nano}
		{Versloot, R. C. A., Straathof, S. A. P., Stouwie, G., Tadema, M. J., \& Maglia, G.}
	\twentyitem{2022}
		{Quantification of Protein Glycosylation Using Nanopores}
		{Nano Letters}
		{Versloot, R. C. A., Lucas, F. L. R., Yakovlieva, L., Tadema, M. J., Zhang, Y., Wood, T. M., Martin, N. I., Marrink, S. J., Walvoort, M. T. C., \& Maglia, G.}
%	\twentyitem{2022}
%		{Unbiased Data Analysis for the Parameterization of Fast Translocation Events through Nanopores}
%		{ACS Omega}
%		{Lucas, F. L. R., Willems, K., Tadema, M. J., Tych, K. M., Maglia, G., \& Wloka, C.}
\end{twenty}


%----------------------------------------------------------------------------------------
%	 AWARDS
%----------------------------------------------------------------------------------------

%\section{Awards}
%
%\begin{twentyshort} % Environment for a short list with no descriptions
%	\twentyitemshort{1987}{All-Time Best Fantasy Novel.}
%	\twentyitemshort{1998}{All-Time Best Fantasy Novel before 1990.}
%	%\twentyitemshort{<dates>}{<title/description>}
%\end{twentyshort}

%----------------------------------------------------------------------------------------
%	 EXPERIENCE
%----------------------------------------------------------------------------------------

\section{Experience}

\begin{twenty} % Environment for a list with descriptions
	\twentyitem{2020-2022}
		{Programming for Life Sciences (Bachelor course)}
		{Teacher}
		{I taught an introductory Python course for Life Science bachelor students three times. During the course the students learn the basics of Python and programming but also how to work with and visualize data using Numpy and Matplotlib.}
	\twentyitem{2019}
		{Bioinformatics (Bachelor course)}
		{Teaching Assistant}
		{At the end of my Master's programme I was a teaching assistant for the Bioinformatics Bachelor course.}
	\twentyitem{2018}
		{IGEM}
		{Student competition}
		{The International Genetically Engineered Machine competition (IGEM) is an international student competition for synthetic biology. Our project "StyGreen" was nominated for best in track.}
	%\twentyitem{<dates>}{<title>}{<location>}{<description>}
\end{twenty}

%----------------------------------------------------------------------------------------
%	 OTHER INFORMATION
%----------------------------------------------------------------------------------------

%\section{Other information}
%
%\subsection{Review}
%
%Alice approaches Wonderland as an anthropologist, but maintains a strong sense of noblesse oblige that comes with her class status. She has confidence in her social position, education, and the Victorian virtue of good manners. Alice has a feeling of entitlement, particularly when comparing herself to Mabel, whom she declares has a ``poky little house," and no toys. Additionally, she flaunts her limited information base with anyone who will listen and becomes increasingly obsessed with the importance of good manners as she deals with the rude creatures of Wonderland. Alice maintains a superior attitude and behaves with solicitous indulgence toward those she believes are less privileged.

%----------------------------------------------------------------------------------------
%	 SECOND PAGE EXAMPLE
%----------------------------------------------------------------------------------------

%\newpage % Start a new page

%\makeprofile % Print the sidebar

%\section{Other information}

%\subsection{Review}

%Alice approaches Wonderland as an anthropologist, but maintains a strong sense of noblesse oblige that comes with her class status. She has confidence in her social position, education, and the Victorian virtue of good manners. Alice has a feeling of entitlement, particularly when comparing herself to Mabel, whom she declares has a ``poky little house," and no toys. Additionally, she flaunts her limited information base with anyone who will listen and becomes increasingly obsessed with the importance of good manners as she deals with the rude creatures of Wonderland. Alice maintains a superior attitude and behaves with solicitous indulgence toward those she believes are less privileged.

%\section{Other information}

%\subsection{Review}

%Alice approaches Wonderland as an anthropologist, but maintains a strong sense of noblesse oblige that comes with her class status. She has confidence in her social position, education, and the Victorian virtue of good manners. Alice has a feeling of entitlement, particularly when comparing herself to Mabel, whom she declares has a ``poky little house," and no toys. Additionally, she flaunts her limited information base with anyone who will listen and becomes increasingly obsessed with the importance of good manners as she deals with the rude creatures of Wonderland. Alice maintains a superior attitude and behaves with solicitous indulgence toward those she believes are less privileged.

%----------------------------------------------------------------------------------------

\end{document} 

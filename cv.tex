%%%%%%%%%%%%%%%%%%%%%%%%%%%%%%%%%%%%%%%%%
% Twenty Seconds Resume/CV
% LaTeX Template
% Version 1.1 (8/1/17)
%
% This template has been downloaded from:
% http://www.LaTeXTemplates.com
%
% Original author:
% Carmine Spagnuolo (cspagnuolo@unisa.it) with major modifications by 
% Vel (vel@LaTeXTemplates.com)
%
% License:
% The MIT License (see included LICENSE file)
%
%%%%%%%%%%%%%%%%%%%%%%%%%%%%%%%%%%%%%%%%%

%----------------------------------------------------------------------------------------
%	PACKAGES AND OTHER DOCUMENT CONFIGURATIONS
%----------------------------------------------------------------------------------------

\documentclass[letterpaper]{twentysecondcv} % a4paper for A4

%----------------------------------------------------------------------------------------
%	 PERSONAL INFORMATION
%----------------------------------------------------------------------------------------

% If you don't need one or more of the below, just remove the content leaving the command, e.g. \cvnumberphone{}

\profilepic{me.jpg} % Profile picture

\cvname{Matthijs Tadema} % Your name
\cvjobtitle{Computational Biologist} % Job title/career

\cvdate{} % Date of birth
\cvaddress{} % Short address/location, use \newline if more than 1 line is required
\cvnumberphone{+31 6 23 51 29 54} % Phone number
\cvsite{linkedin.com/in/mjtadema}
\cvmail{m.j.tadema@pm.me} % Email address

%----------------------------------------------------------------------------------------

\begin{document}

%----------------------------------------------------------------------------------------
%	 ABOUT ME
%----------------------------------------------------------------------------------------

%%bullets:

%Like working with others/social environment
%Enjoy building things 
%In my free time I tinker with linux (servers)
%Autodidact/enjoy learning
%multidisciplinary

\aboutme{

In my free time I like to go cycling and tinker with my Linux server where I self-host a number of web-services for personal use.
} % To have no About Me section, just remove all the text and leave \aboutme{}

%----------------------------------------------------------------------------------------
%	 SKILLS
%----------------------------------------------------------------------------------------

% Skill bar section, each skill must have a value between 0 an 6 (float)
%\skills{{pursuer of rabbits/5.8},{good manners/4},{outgoing/4.3},{polite/4},{Java/0.01}}
\skills{{Python/6},{Bash/5.5},{GROMACS/5.5},{Linux/6},{\LaTeX/5}}

%------------------------------------------------

% Skill text section, each skill must have a value between 0 an 6
\skillstext{{R/2},{C\textbackslash{}C++/1},{Docker/3},{PLUMED/3}}

% Languages

\languages{{Dutch/native},{English/C2}}

%----------------------------------------------------------------------------------------

\makeprofile % Print the sidebar

%----------------------------------------------------------------------------------------
%	 INTERESTS
%----------------------------------------------------------------------------------------

%\section{Interests}

%Molecular simulation, bioinformatics, data science, open source software


%----------------------------------------------------------------------------------------
%	 EXPERIENCE
%----------------------------------------------------------------------------------------

\section{Experience}

\begin{twenty} % Environment for a list with descriptions
	\twentyitem{since 2020}
		{Ph.D. {\normalfont in ChemBio and Molecular Dynamics}}
		{Ph.D. Researcher}
		{
		\begin{itemize}
			\item Developed simulation methodology for high-throughput screening of 
		\end{itemize}
		}
		%{I did my Ph.D. in the Chemical Biology group, bringing together experimental and computational skills I gained during my M.Sc. research in Molecular Dynamics. My research required independently learning many new skills and getting familiar with advanced HPC environments. In this multidisciplinary environment I often collaborated with my colleagues to do computational work for them. Additionally I was responsible for the public-facing website of our research group, as well as taking care of computers at the lab and their data.}
		% 
		% Molecular simulations
		% multidisciplinary
		% programming, HPC
		% Molecular biology lab
		% Taking care of IT tasks, backups, website etc
	\twentyitem{2021-2023}
		{Protein catalyst design}
		{Teaching}
		{I taught tutorials on computational enzyme design, one of my specialties, to chemistry bachelor students. Part of the work was creating slides on the topic and (exam) questions for the students to answer using recent scientific literature as a reference.}
	\twentyitem{2020-2022}
		{Programming for Life Sciences (Bachelor course)}
		{Teaching}
		{I taught an introductory Python course for Life Science bachelor students three times. During the course the students learn the basics of Python and programming and how to work with and visualize data using Numpy and Matplotlib. By helping the students in this course I've had to master reading and understanding other people's code in a short amount of time.}
	\twentyitem{2019}
		{Bioinformatics (Bachelor course)}
		{Teaching Assistant}
		{At the end of my Master's programme I was a teaching assistant for the Bioinformatics Bachelor course. During the course I supervised tutorials and helped students with their questions.}
	%\twentyitem{<dates>}{<title>}{<location>}{<description>}
\end{twenty}

%----------------------------------------------------------------------------------------
%	 EDUCATION
%----------------------------------------------------------------------------------------

\section{Education}

\begin{twenty} % Environment for a list with descriptions
	\twentyitem{since 2020}
		{Ph.D. {\normalfont in Chembio and Molecular Dynamics}}
		{University of Groningen}
		{\emph{Computationally aided protein nanopore design and engineering.}}
	\twentyitem{2017-2020}
		{M.Sc. Biomolecular sciences, magna cum laude}
		{University of Groningen}
		{Strong focus on coarse-grained molecular dynamics and computational design techniques of proteins.}
	\twentyitem{2014-2017}
		{B.Sc. Life Science and Techonlogy}
		{University of Groningen}
		{Molecular biology and enzyme engineering.}
	%\twentyitem{<dates>}{<title>}{<location>}{<description>}
\end{twenty}


%----------------------------------------------------------------------------------------
%	 ADDITIONAL TRAINING
%----------------------------------------------------------------------------------------


\subsection{Additional training}

\begin{twenty}
	\twentyitem{2023}
		{Course on academic writing}
		{University of Groningen}
		{During this course I was taught tips and tricks on academic writing style and how to navigate the world of academic publishing.}
	\twentyitem{2022}
		{Advanced \LaTeX course}
		{University of Groningen}
		{This course focused on advanced use of \LaTeX on overleaf. Some tips on using citing in \LaTeX were also discussed.}
	\twentyitem{2022}
		{SimTech summer school on knowledge-driven machine learning and its applications}
		{University of Stuttgart}
		{This week-long summer school was organized by the Stuttgart Center for Simulation Science. It was insightful to study the possibilities (and limitations) of machine learning and deep learning in various fields of simulation science.}
	\twentyitem{2021}
		{MolSim winter school on molecular simulations}
		{University of Amsterdam}
		{With a focus on the fundamentals of atomistic molecular simulations and statistical thermodynamics, this winter  school was an ideal complement to my background in coarse-grained molecular dynamics. Furthermore, some of the lectures on enhanced sampling techniques proved vital to completing my Ph.D. research.}
\end{twenty}

%----------------------------------------------------------------------------------------
%	 EXTRACURRICULAR ACTIVITIES
%----------------------------------------------------------------------------------------

\newpage % Start a new page

\makeprofile % Print the sidebar

\subsection{Extracurricular activities}

\begin{twenty}
	\twentyitem{2023}
		{Party committee}
		{Planning events for colleagues}
		{In the final year of my Ph.D. I was asked to join the party committee we had in our research group and helped to organize various events such as christmas celebrations, group dinners, outings, etc.}
	\twentyitem{2018}
		{IGEM}
		{Student competition}
		{The International Genetically Engineered Machine competition (IGEM) is an international student competition for synthetic biology organized in Boston, USA. Our project "StyGreen" was nominated for best in track. In addition to being part of the molecular biology lab team and providing molecular simulations, I was also in responsible for the team's finances.}
\end{twenty}
		
		
%----------------------------------------------------------------------------------------
%	 PUBLICATIONS
%----------------------------------------------------------------------------------------

%\section{Publications}
%
%%\begin{twentyshort} % Environment for a short list with no descriptions
%%	\twentyitemshort{JACS, 2023}{Seeing the Invisibles: Detection of Peptide Enantiomers, Diastereomers, and Isobaric Ring Formation in Lanthipeptides Using Nanopores}
%%	%\twentyitemshort{<dates>}{<title/description>}
%%\end{twentyshort}
%
%\begin{twenty}
%	\twentyitem{2023}
%		{Translocation of linearized full-length proteins through an engineered nanopore under opposing electrophoretic force}
%		{Nature Biotechnology}
%		{Sauciuc, A., Morozzo della Rocca, B., Tadema, M. J., Chinappi, M., \& Maglia, G.} 
%	\twentyitem{2023}
%		{Seeing the Invisibles: Detection of Peptide Enantiomers, Diastereomers, and Isobaric Ring Formation in Lanthipeptides Using Nanopores}
%		{JACS}
%		{Abraham Versloot, R. C., Arias-Orozco, P., Tadema, M. J., Rudolfus Lucas, F. L., Zhao, X., Marrink, S. J., Kuipers, O. P., \& Maglia, G.}
%	\twentyitem{2022}
%		{$\beta$-Barrel Nanopores with an Acidic–Aromatic Sensing Region Identify Proteinogenic Peptides at Low pH}
%		{ACS Nano}
%		{Versloot, R. C. A., Straathof, S. A. P., Stouwie, G., Tadema, M. J., \& Maglia, G.}
%	\twentyitem{2022}
%		{Quantification of Protein Glycosylation Using Nanopores}
%		{Nano Letters}
%		{Versloot, R. C. A., Lucas, F. L. R., Yakovlieva, L., Tadema, M. J., Zhang, Y., Wood, T. M., Martin, N. I., Marrink, S. J., Walvoort, M. T. C., \& Maglia, G.}
%%	\twentyitem{2022}
%%		{Unbiased Data Analysis for the Parameterization of Fast Translocation Events through Nanopores}
%%		{ACS Omega}
%%		{Lucas, F. L. R., Willems, K., Tadema, M. J., Tych, K. M., Maglia, G., \& Wloka, C.}
%\end{twenty}


%----------------------------------------------------------------------------------------
%	 AWARDS
%----------------------------------------------------------------------------------------

%\section{Awards}
%
%\begin{twentyshort} % Environment for a short list with no descriptions
%	\twentyitemshort{1987}{All-Time Best Fantasy Novel.}
%	\twentyitemshort{1998}{All-Time Best Fantasy Novel before 1990.}
%	%\twentyitemshort{<dates>}{<title/description>}
%\end{twentyshort}


%----------------------------------------------------------------------------------------
%	 OTHER INFORMATION
%----------------------------------------------------------------------------------------

%\section{Other information}
%
%\subsection{Review}
%
%Alice approaches Wonderland as an anthropologist, but maintains a strong sense of noblesse oblige that comes with her class status. She has confidence in her social position, education, and the Victorian virtue of good manners. Alice has a feeling of entitlement, particularly when comparing herself to Mabel, whom she declares has a ``poky little house," and no toys. Additionally, she flaunts her limited information base with anyone who will listen and becomes increasingly obsessed with the importance of good manners as she deals with the rude creatures of Wonderland. Alice maintains a superior attitude and behaves with solicitous indulgence toward those she believes are less privileged.

%----------------------------------------------------------------------------------------
%	 SECOND PAGE EXAMPLE
%----------------------------------------------------------------------------------------

%\newpage % Start a new page

%\makeprofile % Print the sidebar

%\section{Other information}

%\subsection{Review}

%Alice approaches Wonderland as an anthropologist, but maintains a strong sense of noblesse oblige that comes with her class status. She has confidence in her social position, education, and the Victorian virtue of good manners. Alice has a feeling of entitlement, particularly when comparing herself to Mabel, whom she declares has a ``poky little house," and no toys. Additionally, she flaunts her limited information base with anyone who will listen and becomes increasingly obsessed with the importance of good manners as she deals with the rude creatures of Wonderland. Alice maintains a superior attitude and behaves with solicitous indulgence toward those she believes are less privileged.

%\section{Other information}

%\subsection{Review}

%Alice approaches Wonderland as an anthropologist, but maintains a strong sense of noblesse oblige that comes with her class status. She has confidence in her social position, education, and the Victorian virtue of good manners. Alice has a feeling of entitlement, particularly when comparing herself to Mabel, whom she declares has a ``poky little house," and no toys. Additionally, she flaunts her limited information base with anyone who will listen and becomes increasingly obsessed with the importance of good manners as she deals with the rude creatures of Wonderland. Alice maintains a superior attitude and behaves with solicitous indulgence toward those she believes are less privileged.

%----------------------------------------------------------------------------------------

\end{document} 

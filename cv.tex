%%%%%%%%%%%%%%%%%%%%%%%%%%%%%%%%%%%%%%%%%
% Twenty Seconds Resume/CV
% LaTeX Template
% Version 1.1 (8/1/17)
%
% This template has been downloaded from:
% http://www.LaTeXTemplates.com
%
% Original author:
% Carmine Spagnuolo (cspagnuolo@unisa.it) with major modifications by 
% Vel (vel@LaTeXTemplates.com)
%
% License:
% The MIT License (see included LICENSE file)
%
%%%%%%%%%%%%%%%%%%%%%%%%%%%%%%%%%%%%%%%%%

%----------------------------------------------------------------------------------------
%	PACKAGES AND OTHER DOCUMENT CONFIGURATIONS
%----------------------------------------------------------------------------------------

\documentclass[letterpaper]{twentysecondcv} % a4paper for A4

%----------------------------------------------------------------------------------------
%	 PERSONAL INFORMATION
%----------------------------------------------------------------------------------------

% If you don't need one or more of the below, just remove the content leaving the command, e.g. \cvnumberphone{}

\profilepic{me_light.jpg} % Profile picture

\cvname{Matthijs Tadema} % Your name
\cvjobtitle{Computational Biologist} % Job title/career

\cvdate{} % Date of birth
\cvaddress{} % Short address/location, use \newline if more than 1 line is required
\cvnumberphone{+31 6 23 51 29 54} % Phone number
\cvsite{linkedin.com/in/mjtadema}
\cvmail{m.j.tadema@pm.me} % Email address

%----------------------------------------------------------------------------------------

\begin{document}

%----------------------------------------------------------------------------------------
%	 ABOUT ME
%----------------------------------------------------------------------------------------

%%bullets:

%Like working with others/social environment
%Enjoy building things 
%In my free time I tinker with linux (servers)
%Autodidact/enjoy learning
%multidisciplinary

\aboutme{
Multidisciplinary computational biologist with affinity for programming and system administration looking to transition out of academia.
During my Ph.D. I realized I enjoy programming and using my skills to help others more than doing research myself.
Excited to learn new skills and find a more human, collaborative work environment.
} % To have no About Me section, just remove all the text and leave \aboutme{}

%----------------------------------------------------------------------------------------
%	 SKILLS
%----------------------------------------------------------------------------------------

% Skill bar section, each skill must have a value between 0 an 6 (float)
%\skills{{pursuer of rabbits/5.8},{good manners/4},{outgoing/4.3},{polite/4},{Java/0.01}}
\skills{{Python/6},{Bash/5.5},{GROMACS/5.5},{Linux/6},{\LaTeX/5}}

%------------------------------------------------

% Skill text section, each skill must have a value between 0 an 6
\skillstext{{R/2},{C\textbackslash{}C++/1},{Docker/3},{PLUMED/3},{PyMOL/6}}

% Languages

\languages{{Dutch/native},{English/C2}}

%----------------------------------------------------------------------------------------

\makeprofile % Print the sidebar

%----------------------------------------------------------------------------------------
%	 INTERESTS
%----------------------------------------------------------------------------------------

%\section{Interests}

%Molecular simulation, bioinformatics, data science, open source software


%----------------------------------------------------------------------------------------
%	 EXPERIENCE
%----------------------------------------------------------------------------------------

\section{Experience}

\begin{twenty} % Environment for a list with descriptions
	\twentyitem{since 2020}
		{Ph.D. {\normalfont in ChemBio and Molecular Dynamics}}
		{Ph.D. Researcher}
		{\begin{itemize}
			\item Developed simulation methodology for high-throughput screening of nanopore translocation energetics
			\item Protein engineering and expression
			\item Multidisciplinary environment
			\item Gained extensive Python and Bash programming experience
			\item Became deeply familiar with Linux systems and HPC environments
		\end{itemize}}
		
		%{I did my Ph.D. in the Chemical Biology group, bringing together experimental and computational skills I gained during my M.Sc. research in Molecular Dynamics. My research required independently learning many new skills and getting familiar with advanced HPC environments. In this multidisciplinary environment I often collaborated with my colleagues to do computational work for them. Additionally I was responsible for the public-facing website of our research group, as well as taking care of computers at the lab and their data.}
		% 
		% Molecular simulations
		% multidisciplinary
		% programming, HPC
		% Molecular biology lab
		% Taking care of IT tasks, backups, website etc
	\twentyitem{2021-2023}
		{B.Sc. course Protein catalyst design}
		{Teaching}
		{\begin{itemize}
			\item Taught tutorials about computational enzyme design
			\item Created slides on the topic and (exam) questions for the students to answer using recent scientific literature as a reference
		\end{itemize}}
	\twentyitem{2020-2022}
		{B.Sc. course Programming for Life Sciences}
		{Teaching}
		{\begin{itemize}
			\item Data processing and visualization using NumPy and Matplotlib
			\item Reading and debugging other people's codes
		\end{itemize}}
	\twentyitem{2019}
		{B.Sc. course Bioinformatics}
		{Teaching Assistant}
		{\begin{itemize}
			\item Supervising students during tutorials and answering questions
		\end{itemize}}		
		%{At the end of my Master's programme I was a teaching assistant for the Bioinformatics Bachelor course. During the course I supervised tutorials and helped students with their questions.}
	%\twentyitem{<dates>}{<title>}{<location>}{<description>}
\end{twenty}

%----------------------------------------------------------------------------------------
%	 EDUCATION
%----------------------------------------------------------------------------------------

\section{Education}

\begin{twenty} % Environment for a list with descriptions
	\twentyitem{since 2020}
		{Ph.D. {\normalfont in Chembio and Molecular Dynamics}}
		{University of Groningen}
		{\emph{Molecular simulation aided protein nanopore design and engineering.}}
	\twentyitem{2017-2020}
		{M.Sc. Biomolecular sciences, magna cum laude}
		{University of Groningen}
		{Molecular biology with a strong focus on coarse-grained molecular dynamics and computational design techniques of proteins.}
	\twentyitem{2014-2017}
		{B.Sc. Life Science and Techonlogy}
		{University of Groningen}
		{Molecular biology and enzyme engineering.}
	%\twentyitem{<dates>}{<title>}{<location>}{<description>}
\end{twenty}


%----------------------------------------------------------------------------------------
%	 ADDITIONAL TRAINING
%----------------------------------------------------------------------------------------


\subsection{Additional training}

\begin{twenty}
	\twentyitem{2023}
		{Course on academic writing}
		{University of Groningen}
		{\begin{itemize}
			\item Learned academic writing style and publishing.
		\end{itemize}}
	\twentyitem{2022}
		{Advanced \LaTeX course}
		{University of Groningen}
		{\begin{itemize}
			\item Use of \LaTeX and overleaf
			\item Citing and reference management in \LaTeX
		\end{itemize}}
	\twentyitem{2022}
		{SimTech summer school on knowledge-driven machine learning and its applications}
		{University of Stuttgart}
		{\begin{itemize}
			\item Possibilities (and limitations) of machine learning and deep learning in various fields of simulation science
		\end{itemize}}
	\twentyitem{2021}
		{MolSim winter school on molecular simulations}
		{University of Amsterdam}
		{\begin{itemize}
			\item Fundamentals of atomistic molecular simulations and statistical thermodynamics
			\item Introduction to enhanced sampling techniques and path sampling techniques
		\end{itemize}}
\end{twenty}

%----------------------------------------------------------------------------------------
%	 EXTRACURRICULAR ACTIVITIES
%----------------------------------------------------------------------------------------

\newpage % Start a new page

\makeprofile % Print the sidebar

\subsection{Extracurricular activities}

\begin{twenty}
	\twentyitem{2023}
		{Party committee}
		{Planning events for colleagues}
		{\begin{itemize}
			\item Helped organize various events such as Christmas celebrations, group dinners, outings, etc.
		\end{itemize}}
	\twentyitem{2018}
		{IGEM}
		{International student competition}
		{\begin{itemize}
			\item Student competition in synthetic biology in the US
			\item Provided molecular dynamics simulations
			\item Coordinated experimental team
			\item Responsible for finances
			\item Project nominated for "best in track"
		\end{itemize}}
\end{twenty}
		
		
%----------------------------------------------------------------------------------------
%	 PUBLICATIONS
%----------------------------------------------------------------------------------------

%\section{Publications}
%
%%\begin{twentyshort} % Environment for a short list with no descriptions
%%	\twentyitemshort{JACS, 2023}{Seeing the Invisibles: Detection of Peptide Enantiomers, Diastereomers, and Isobaric Ring Formation in Lanthipeptides Using Nanopores}
%%	%\twentyitemshort{<dates>}{<title/description>}
%%\end{twentyshort}
%
%\begin{twenty}
%	\twentyitem{2023}
%		{Translocation of linearized full-length proteins through an engineered nanopore under opposing electrophoretic force}
%		{Nature Biotechnology}
%		{Sauciuc, A., Morozzo della Rocca, B., Tadema, M. J., Chinappi, M., \& Maglia, G.} 
%	\twentyitem{2023}
%		{Seeing the Invisibles: Detection of Peptide Enantiomers, Diastereomers, and Isobaric Ring Formation in Lanthipeptides Using Nanopores}
%		{JACS}
%		{Abraham Versloot, R. C., Arias-Orozco, P., Tadema, M. J., Rudolfus Lucas, F. L., Zhao, X., Marrink, S. J., Kuipers, O. P., \& Maglia, G.}
%	\twentyitem{2022}
%		{$\beta$-Barrel Nanopores with an Acidic–Aromatic Sensing Region Identify Proteinogenic Peptides at Low pH}
%		{ACS Nano}
%		{Versloot, R. C. A., Straathof, S. A. P., Stouwie, G., Tadema, M. J., \& Maglia, G.}
%	\twentyitem{2022}
%		{Quantification of Protein Glycosylation Using Nanopores}
%		{Nano Letters}
%		{Versloot, R. C. A., Lucas, F. L. R., Yakovlieva, L., Tadema, M. J., Zhang, Y., Wood, T. M., Martin, N. I., Marrink, S. J., Walvoort, M. T. C., \& Maglia, G.}
%%	\twentyitem{2022}
%%		{Unbiased Data Analysis for the Parameterization of Fast Translocation Events through Nanopores}
%%		{ACS Omega}
%%		{Lucas, F. L. R., Willems, K., Tadema, M. J., Tych, K. M., Maglia, G., \& Wloka, C.}
%\end{twenty}


%----------------------------------------------------------------------------------------
%	 AWARDS
%----------------------------------------------------------------------------------------

%\section{Awards}
%
%\begin{twentyshort} % Environment for a short list with no descriptions
%	\twentyitemshort{1987}{All-Time Best Fantasy Novel.}
%	\twentyitemshort{1998}{All-Time Best Fantasy Novel before 1990.}
%	%\twentyitemshort{<dates>}{<title/description>}
%\end{twentyshort}


%----------------------------------------------------------------------------------------
%	 OTHER INFORMATION
%----------------------------------------------------------------------------------------

%\section{Other information}
%
%\subsection{Review}
%
%Alice approaches Wonderland as an anthropologist, but maintains a strong sense of noblesse oblige that comes with her class status. She has confidence in her social position, education, and the Victorian virtue of good manners. Alice has a feeling of entitlement, particularly when comparing herself to Mabel, whom she declares has a ``poky little house," and no toys. Additionally, she flaunts her limited information base with anyone who will listen and becomes increasingly obsessed with the importance of good manners as she deals with the rude creatures of Wonderland. Alice maintains a superior attitude and behaves with solicitous indulgence toward those she believes are less privileged.

%----------------------------------------------------------------------------------------
%	 SECOND PAGE EXAMPLE
%----------------------------------------------------------------------------------------

%\newpage % Start a new page

%\makeprofile % Print the sidebar

%\section{Other information}

%\subsection{Review}

%Alice approaches Wonderland as an anthropologist, but maintains a strong sense of noblesse oblige that comes with her class status. She has confidence in her social position, education, and the Victorian virtue of good manners. Alice has a feeling of entitlement, particularly when comparing herself to Mabel, whom she declares has a ``poky little house," and no toys. Additionally, she flaunts her limited information base with anyone who will listen and becomes increasingly obsessed with the importance of good manners as she deals with the rude creatures of Wonderland. Alice maintains a superior attitude and behaves with solicitous indulgence toward those she believes are less privileged.

%\section{Other information}

%\subsection{Review}

%Alice approaches Wonderland as an anthropologist, but maintains a strong sense of noblesse oblige that comes with her class status. She has confidence in her social position, education, and the Victorian virtue of good manners. Alice has a feeling of entitlement, particularly when comparing herself to Mabel, whom she declares has a ``poky little house," and no toys. Additionally, she flaunts her limited information base with anyone who will listen and becomes increasingly obsessed with the importance of good manners as she deals with the rude creatures of Wonderland. Alice maintains a superior attitude and behaves with solicitous indulgence toward those she believes are less privileged.

%----------------------------------------------------------------------------------------

\end{document} 
